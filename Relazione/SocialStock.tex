 \documentclass[11pt,a4paper]{article}
\usepackage[utf8]{inputenc}
\usepackage[a4paper]{geometry}
\usepackage{amsmath}
\usepackage{amsfonts}
\usepackage{amssymb}
\usepackage{amsthm}
\usepackage{newlfont}
\usepackage[british,UKenglish,USenglish,english,american]{babel}
\usepackage{rotating}
\usepackage{subfigure}
\usepackage{lscape}
\usepackage[bf, scriptsize]{caption}
\usepackage{multirow}
\usepackage{longtable}
\hyphenation{sil-la-ba-zio-ne pa-ren-te-si}
\hyphenation{Low-din}
\usepackage{bbm}
\usepackage {amsmath, amssymb} 
\usepackage{graphicx}
\usepackage{titling}
\usepackage{eurosym}
\usepackage{verbatim}
\usepackage{xcolor}
\usepackage{alltt}


\author{Federico Cinus, Federico Delussu, Nicola Pacella}
\author{Master’s Students in Physics of Complex System}
\title{ \LARGE{UNIVERSIT\`{A} DEGLI STUDI DI TORINO} 
\\
MultiAgent Systems Course A.A. 2017/2018 Prof. Marco Maggiora
\\
 \textbf{Influence in social network}}

\begin{document}
\date{}
\maketitle
\bigskip


%---------------------------------------1-------------------------------------%
\tableofcontents
%---------------------------------------2-------------------------------------%

\newpage 
\section{Introduction}

\section{The Model}

\begin{comment}

SCHEMA DEL MODELLO

la classe user

	gli attributi personali dello user : opinion vector (V) /inclination (I) /percdiff (p) /degree (d)  
							       
	gli attributi statici : opinion range (R) / gamma (G) / userNumber (N) / companies (C) 
	 
		opinion vector -> tante entrate quante le companies 
			ciascun entrata può assumere un valore fra 0 e l'opinion range R
						
\end{comment}

The initial world's configuration is given by the instantiation of $N$ Users. 
The Users live in a world with $C$ companies and are provided with a set of opinions, one for each company. 
The specific opinion is a positive integer which ranges from $0$ to $R-1$, let $R$ denote the opinion range.
Each User is denoted by an inclination $I \in \{-1,0,1\}$ which represents its average opinion along the companies. The $-1, 0,$ and $1$ values denote respectively a "bad", "neutral" and "good" average opinion.  

The simulation is made up by a series of $D$ temporal steps which we call days. 
On the first day the Users' opinions are randomly intialized. 
On each day the Users influence each other by one-to-one interactions in which opinions can be exchanged.
Each User has a degree $k$ which is the number of Users with which he can interact during the day.








\subsection{The User Class}
	





\section{Graphs and results}

\section{Error estimation}

\section{Conclusions}

\section{Altre cose che non so dove mettere}


\begin{comment}

\begin{center}
\includegraphics[width = 11cm, height = 11cm]{exe1_smd.pdf}
\end{center}
\newpage 

\end{comment}


\end{document}
