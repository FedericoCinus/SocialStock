 \documentclass[11pt,a4paper]{article}
\usepackage[utf8]{inputenc}
\usepackage[a4paper]{geometry}
\usepackage{amsmath}
\usepackage{amsfonts}
\usepackage{amssymb}
\usepackage{amsthm}
\usepackage{newlfont}
\usepackage[british,UKenglish,USenglish,english,american]{babel}
\usepackage{rotating}
\usepackage{subfigure}
\usepackage{lscape}
\usepackage[bf, scriptsize]{caption}
\usepackage{multirow}
\usepackage{longtable}
\hyphenation{sil-la-ba-zio-ne pa-ren-te-si}
\hyphenation{Low-din}
\usepackage{bbm}
\usepackage {amsmath, amssymb} 
\usepackage{graphicx}
\usepackage{titling}
\usepackage{eurosym}
\usepackage{verbatim}
\usepackage{xcolor}
\usepackage{alltt}


\author{Federico Cinus, Federico Delussu, Nicola Pacella}
\author{Master’s Students in Physics of Complex System}
\title{ \LARGE{UNIVERSIT\`{A} DEGLI STUDI DI TORINO} 
\\
MultiAgent Systems Course A.A. 2017/2018 Prof. Marco Maggiora
\\
 \textbf{Influence in social network}}

\begin{document}
\date{}
\maketitle
\bigskip


%---------------------------------------1-------------------------------------%
\tableofcontents
%---------------------------------------2-------------------------------------%

\newpage 
\section{Introduction}

\section{The Model}

\subsection{Users and Companies}

The initial world's configuration is given by the instantiation of $N$ Users. 
The Users live in a world with $C$ companies (which are indexed from $0$ to $C-1$) and are provided with a set of opinions, one for each company.

An opinion is a positive integer which ranges from $0$ to $R-1$, let $R$ denote the opinion range. For each User the opinions are stored within a vector $ \mathbf{v} \in \mathbb{N}^C$ which we denote as the opinion vector, the entry $  \mathbf{v}_c $ is the opinion on the $c$-th company.

Each User has an inclination $I \in \{-1,0,1\}$ whose possible values denote respectively a "bad", "neutral" and "good" average opinion across the companies, its computation is described with further detail in the following pages.
  
The simulation is made up by a series of $D$ temporal steps which we call days. 
On first day the Users' opinions are randomly intialized. 
On each day the Users influence each other with one-to-one interactions in which opinions can be exchanged.

\subsection{Interaction}

Each User has a degree $k$ which is the number of Users to whom he can send messages during the day.

Let's consider an interacting couple and call the two Users A and B having respectively degree $k_A$ and 
$k_B$, the interaction consists in three steps :
\begin{itemize}
	\item[1] A sends a message to B
	\item[] The message contains information on A's opinion vector $\mathbf{v}_A$ and degree $k_A$  
	\item[2] B receives A's message
	\item[] B compares its opinion vector $\mathbf{v}_B$ with $\mathbf{v}_A$, given the subset of companies on which the two Users' opinion are different, a company's index is extracted according to a uniform distribution. If $\mathbf{v}_A$ and $\mathbf{v}_B$ are equal the interaction stops. 
	\item[3] Influence	
	\item[] Let $c$ denote the extracted index, two possible events may occur: 
	\begin{itemize}
		\item The opinion on $c$-th company remains unchanged for B while A changes it to B's opinion. This event occurs with probability $k_B/(k_A + k_B)$.
		\item The opinion on $c$-th company remains unchanged for A while B changes it to A's opinion. This event occurs with probability $k_A/(k_A + k_B)$.
		\item[] Basically each User can influence the other on a differing opinion with a probability proportional to its degree. 	 	
	\end{itemize} 		
\end{itemize}

\subsection{Inclination Computation}

\begin{comment}
Io metterei il calcolo dell'inclination in appendice 
\end{comment}

The Inclination for each User is computed by considering the average of the opinions across the companies.

Let's consider an opinion vector $\mathbf{v}$ with entries $\mathbf{v}_c$ ($ c = 1, .., C $).
Each entry $\mathbf{v}_c$ is drawn uniformly from the discrete set $\{ 0, 1,.., R-1 \}$.
Recalling that the uniform discrete distribution $\mathcal{U}\{a,b\}$ has mean $\frac{a+b}{2}$ and variance ${\frac  {(b-a+1)^{2}-1}{12}}$ we have that $\mathcal{U}\{0,R-1\}$ has mean $\frac{R-1}{2}$ and variance ${\frac  {R^{2}-1}{12}}$

Let $S$ be the average of $\mathbf{v}$ entries:
$$ S = \sum_{c=1}^{C} \frac{ \mathbf{v}_c }{ C } $$ 

For the central limit theorem, for $C \rightarrow \infty$ the random variable $S$ will follow a normal distribution with mean $\frac{R-1}{2}$ and variance ${\frac  {R^{2}-1}{12C}}$.

In order to assess wether the inclination $I=-1$ we check if the value of $S$ lies under the quantile of order $1/3$ of the normal distribution ${\mathcal {N}}( \frac{R-1}{2} , {\frac  {R^{2}-1}{12C}})$. If $I=0$ then $S$ is between the $1/3$ and $2/3$ quantile, and $I=1$ if $S$ is over the $2/3$ quantile.

The $1/3$ quantile of the standard normal ${\mathcal {N}}(0,1)$ is $q_{1/3}^{(S)} = -0.4399132$ and by simmetry $q_{2/3}^{(S)} = - q_{1/3}^{(S)} $. The $1/3$ quantile of ${\mathcal {N}}(\mu ,\sigma ^{2})$ is $q_{1/3} = \mu + \sigma q_{1/3}^{(S)}$ and the same applies for the $2/3$ quantile. 

So, in summary, given the User's opinion vector $\mathbf{v}$ we compute its inclination according to the average of components $S$: 

$$  I :={\begin{cases}-1&{\text{if }} S < \frac{R-1}{2} +  q_{1/3}^{(S)} \frac  {R^{2}-1}{12C},
\\1&{\text{if }}  S > \frac{R-1}{2} +  q_{2/3}^{(S)} \frac  {R^{2}-1}{12C},
\\0&{\text{if else }} \end{cases}}  $$
  

\subsection{The User Class}
	





\section{Graphs and results}

\section{Error estimation}

\section{Conclusions}

\section{Altre cose che non so dove mettere}


\begin{comment}

\begin{center}
\includegraphics[width = 11cm, height = 11cm]{exe1_smd.pdf}
\end{center}
\newpage 

\end{comment}


\end{document}
